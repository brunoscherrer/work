\documentclass{article}
\usepackage{fullpage}
\usepackage[utf8x]{inputenc}
\usepackage{amssymb}
\usepackage{amsthm}
\usepackage{amsmath}
\usepackage{hyperref}
\usepackage{cleveref}
\usepackage{autonum}
\usepackage{dsfont}
\usepackage{natbib}


\newtheorem{theorem}{Theorem}
\newtheorem{lemma}{Lemma}
\newtheorem{remark}{Remark}


\title{A polynomial algorithm for the parity game}

\author{Bruno Scherrer\footnote{INRIA, Universit\'e de Lorraine, bruno.scherrer@inria.fr}}


\def\1{{\mathds 1}}
\def\N{\mathbb N}
\def\R{\mathbb R}
\def\G{{\cal G}}

\begin{document}
\maketitle

\begin{abstract}
  ...
\end{abstract}


Given an arena $\G=(X=[1,n]=X_1 \sqcup X_2,E=[1,m],\Omega)$ with $(n,m) \in \N^2$, a \emph{parity game} is a game played by two players, Odd and Even.
$(X,E)$ is a graph without any dead ends: $X$ is a set of $n$ nodes and $E$ a set of $m$ directed edges such that each node has a least one outgoing edge. 
The set of nodes  $X$ is partitioned into a set of states $X_1$ belonging to Odd and a set of nodes $X_2$ belonging to Even. $\Omega:X \to [1,d]$, known as a priority function, assigns an integer label to each node of the graph.
A play is an infinitely long trajectory $(v_0,v_1,\dots)$ generated from some starting state $v_0$: at any time step $t$, the player to which the node $v_t$ chooses $v_{t+1}$ among the adjacent nodes from $v_t$ (following any of the outgoing edges of $E$ starting from $v_t$). The winner of the game is decided from the infinte sequence of priorities $(\Omega(v_0),\Omega(v_1),\dots$ occuring through the play: if the highest priority occurring infinitely often is odd, then Odd wins. Otherwise (if it is even), Even wins. 

A \emph{mean-payoff game} is a game played by two players, Max and Min, on a arena $\G=(X=[1,n]=X_1 \sqcup X_2, E=[1,m],w)$ similar to that of a parity game ; the only difference is that the priority function is replaced by a cost function $w:X \to [-W,W]$ where $W \in \N$. The dynamics of the game is the same as above. On potential plays $(v_0,v_1,\dots)$ induced by the players' choices, Max wants to maximize $\liminf_{t \to \infty}\frac{1}{t} \sum_{i=1}^t w(v_i)$ while Min wants to minimize $\limsup_{t \to \infty}\frac{1}{t} \sum_{i=1}^t g(v_i)$. \citet{ehrenfeucht79} have shown that for each starting node $v_0$,  such a game has a value $\nu(v_0)$, the optimal mean-payoff from $v_0$, such Max has a strategy to ensure that $\limsup_{t \to \infty}\frac{1}{t} \sum_{i=0}^t w(v_i) \ge \nu(v_0)$ and Min has a strategy to ensure that $\limsup_{t \to \infty}\frac{1}{t} \sum_{i=0}^t g(v_i) \le \nu(v_0)$.

For both games, it is known (cf. \citet{zielonka98} and \citet{ehrenfeucht79}) that there exist optimal strategies that are positional (i.e. that are mapping from nodes to outgoing edges). In particular, when both players follow these positional strategies from some state $v_0$, the play follows a (potentially empty) path followed by an infinitely-repeated cycle with optimal maximum parity of optimal mean payoff, in other words an optimal cycle $(v^*_1,\dots,v^*_c)$.

\paragraph{Puri's reduction}

As shown by \citet{puri96}, any parity game $\G=(X=[1,n]=X_1 \sqcup X_2,E=[1,m],\Omega)$ can be reduced to a mean-payoff game $\G'=(X=[1,n]=X_1 \sqcup X_2,E=[1,m],w)$, by choosing the weight function as follows
\begin{align}
  \forall v,~~~ w(v) = (-K)^{\Omega(v)},
\end{align}
for some sufficiently big integer.
Indeed, consider an optimal cycle $(v^*_1,\dots,v^*_c)$ (using the above-mentionned positional strategies) with maximal parity $\Omega^*$. If $\Omega^*$ is even then, by choosing $K=n$, we have
\begin{align}
 0 <  K^{\Omega^*} - (n-1) K^{\Omega^*-1} \le  \sum_{i=1}^{c} (-K)^{\Omega^*(v^*_i)} \le n K^{\Omega^*}.
\end{align}
If $\Omega^*$ is odd, we similarly have:
\begin{align}
0 > -K^{\Omega^*} + (n-1) K^{\Omega^*-1} \ge   \sum_{i=1}^{c} (-K)^{\Omega^*(v^*_i)} \ge -n K^{\Omega^*}.
\end{align}
In other words, from any starting node $v_0$, Even (resp. Odd) wins the parity game if the value of the mean-payoff game $\nu(v_0)$ is positive (resp. negative). Furthermore, an optimal pair of strategies for the mean-payoff game is optimal in the parity game.

\section{Extended parity game}

We shall consider a slight variation of Puri's reduction where instead of $K=n$, we take $K=n+1$. Doing so, given an optimal cycle $(v^*_1,\dots,v^*_c)$, when $\Omega^*$ is even, by using the fact that $(n+1)^2-(n-1)(n+1)=2(n+1)$, we can see that
\begin{align}
   2 K^{\Omega^*} \le  K^{\Omega^*} - (n-1) K^{\Omega^*-1} \le  \sum_{i=1}^{c} (-K)^{\Omega^*(v^*_i)} \le n K^{\Omega^*}.
\end{align}
Similarly, when $\Omega^*$ is odd, we have
\begin{align}
  - 2 K^{\Omega^*} \le  -K^{\Omega^*} + (n-1) K^{\Omega^*-1} \ge \sum_{i=1}^{c} (-K)^{\Omega^*(v^*_i)} \ge -n K^{\Omega^*}.
\end{align}
Through this reduction, the parity game is made more precise: when each player loses, he tries to make the parity with which the game is won as small as possible.

Finally, we shall consider an extension of the parity game where some nodes can have a neutral priority, in which case the cost in the mean-payoff game is $0$.

\section{0-1 mean payoff game}

\section{The algorithm}


\bibliographystyle{plainnat}
\bibliography{biblio.bib} 

\end{document}
