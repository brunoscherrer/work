\documentclass{article}
\usepackage{fullpage}
\usepackage[utf8x]{inputenc}
\usepackage{amssymb}
\usepackage{amsthm}
\usepackage{amsmath}
\usepackage{hyperref}
\usepackage{cleveref}
\usepackage{autonum}
\usepackage{dsfont}
\usepackage{natbib}


\newtheorem{theorem}{Theorem}
\newtheorem{lemma}{Lemma}
\newtheorem{remark}{Remark}


\title{A polynomial algorithm for the parity game}

\author{Bruno Scherrer\footnote{INRIA, Universit\'e de Lorraine, bruno.scherrer@inria.fr}}


\def\1{{\mathds 1}}
\def\N{\mathbb N}
\def\R{\mathbb R}
\def\G{{\cal G}}

\begin{document}
\maketitle

\begin{abstract}
  ...
\end{abstract}


\paragraph{Parity and mean-payoff games}

Given an arena $\G=(X=[1,n]=X_0 \sqcup X_1,E=[1,m],p)$ with $(n,m) \in \N^2$, a \emph{parity game} is a game played by two players, ODD and EVEN.
$(X,E)$ is a directed graph. $X$ is a set of $n$ nodes and $E$ a set of $m$ directed edges such that each node has a least one outgoing edge. 
The set of nodes  $X$ is partitioned into a set of states $X_1$ belonging to ODD and a set of nodes $X_0$ belonging to EVEN. $p:X \to [1,d]$, known as a priority function, assigns an integer label to each node of the graph.
A play is an infinitely long trajectory $(x_0,x_1,\dots)$ generated from some starting state $x_0$: at any time step $t$, the player to which the node $x_t$ belongs chooses $x_{t+1}$ among the adjacent nodes from $x_t$ (following any of the outgoing edges of $E$ starting from $x_t$). The winner of the game is decided from the infinte sequence of priorities $(p(x_0),p(x_1),\dots)$ occuring through the play: if the highest priority occurring infinitely often is odd, then ODD wins. Otherwise (if it is even), EVEN wins. 

A \emph{mean-payoff game} is a game played by two players, Max and Min, on a arena $\G=(X=[1,n]=X_1 \sqcup X_0, E=[1,m],w)$ similar to that of a parity game ; the only difference is that the priority function is replaced by a cost function $w:X \to [-W,W]$ where $W \in \N$. The dynamics of the game is the same as above. On potential plays $(x_0,x_1,\dots)$ induced by the players' choices, Max wants to maximize $\liminf_{t \to \infty}\frac{1}{t} \sum_{i=1}^t w(x_i)$ while Min wants to minimize $\limsup_{t \to \infty}\frac{1}{t} \sum_{i=1}^t g(x_i)$. \citet{ehrenfeucht79} have shown that for each starting node $x_0$,  such a game has a value $\nu(x_0)$, the optimal mean-payoff from $x_0$, such Max has a strategy to ensure that $\limsup_{t \to \infty}\frac{1}{t} \sum_{i=0}^t w(x_i) \ge \nu(x_0)$ and Min has a strategy to ensure that $\limsup_{t \to \infty}\frac{1}{t} \sum_{i=0}^t g(x_i) \le \nu(x_0)$.

For both games, it is known (cf. \citet{zielonka98} and \citet{ehrenfeucht79}) that there exist optimal strategies that are positional (i.e. that are mapping from nodes to outgoing edges). In particular, when both players follow these positional strategies from some state $x_0$, the play follows a (potentially empty) path followed by an infinitely-repeated cycle, in other words an optimal cycle $(x^*_1,\dots,x^*_c)$.

\section{The optimal priority of a parity game}

\citet{puri96} has introduced the following reduction of any parity game $\G=(X=[1,n]=X_1 \sqcup X_0,E=[1,m],p)$ to a mean-payoff game $\G'=(X=[1,n]=X_1 \sqcup X_0,E=[1,m],w)$ that involved the exact same graph $(X,E)$ and the weight function:
\begin{align}
  \forall x,~~~ w(x) = (-n)^{p(x)}.
\end{align}
Indeed, consider an optimal cycle $(x^*_1,\dots,x^*_c)$ of the mean-payoff game (using the above-mentionned positional strategies). Let $p=\max_{1 \le i \le c} p(x^*_i)$ be the maximal priority obtained in this cycle. If $p$ is even then
\begin{align}
 0 <  n^{p} - (n-1) n^{p-1} \le  \sum_{i=1}^{c} (-n)^{p(x^*_i)}.
\end{align}
If $p$ is odd, we similarly have:
\begin{align}
0 > -n^{p} + (n-1) n^{p-1} \ge   \sum_{i=1}^{c} (-n)^{p(x^*_i)}.
\end{align}
As a consequence, from any starting node $x_0$, EVEN (resp. ODD) wins the parity game if the value $v(x_0)$ of the mean-payoff game  is positive (resp. negative). Furthermore, an optimal pair of strategies for the mean-payoff game is also optimal for the parity game (note that the opposite is in general not true).

We shall consider a slight variation of Puri's reduction that consists in choosing the alternative weight function:
\begin{align}
  \forall x,~~~ w(x) = (-K)^{p(x)}
\end{align}
with any $K$ such that $K-(n-1) > n^2$
(for instance one may take $K=(n+1)^2$).

Consider an optimal cycle $(x^*_1,\dots,x^*_c)$ of this mean-payoff game. Let $p=\max_{1 \le i \le c} p(x^*_i)$ be the maximal priority obtained in this cycle. If $p$ is even then
\begin{align}
n^2 K^{p-1} <  (K-(n-1))K^{p-1} =   K^p - (n-1) K^{p-1}  \le \sum_{i=1}^{c} (-K)^{p(x^*_i)} \le n K^p,
\end{align}
and the value $v(x_0)$ from any state $x_0$ that reaches this cycle is such that
\begin{align}
  n K^{p-1} \le \frac{n^2 K^{p-1}}{c} < v(x_0) \le \frac{n}{c} K^p \le n K^p. 
\end{align}
When $p$ is odd, we have
\begin{align}
  - n^2 K^{p-1} > -(K-(n-1))K^{p-1} =   -K^p + (n-1) K^{p-1} \ge \sum_{i=1}^{c} (-K)^{p(x^*_i)} \ge -n K^p,
\end{align}
and the value $v(x_0)$ from any state $x_0$ that reaches this cycle is such that
\begin{align}
  -n K^{p-1} \ge \frac{-n^2 K^{p-1}}{c} > v(x_0) \ge -\frac{n}{c}K^p \ge - n K^p. 
\end{align}
From any starting node $x_0$, we shall say that the \emph{optimal priority $p$} is the unique value $p$ such that $n K^{p-1} < v(x_0) \le n K^p$ or $-n K^{p-1} > v(x_0) \ge -n K^p$. If this priority is even (resp. odd), EVEN (resp. ODD) wins the parity game from $x_0$.

Through this slightly modified reduction, one makes the parity game more precise: from any starting state, each player that loses tries to make the priority with which the game is won by the other player as low as possible. 


\section{An algorithm for computing the optimal parity}

We now describe a recursive algorithm that computes \emph{optimal parity} $p_*(x)$ for all states $x$.
\paragraph{Terminal condition:} If the parity game only contains one priority $p$, then we know that for all states, the optimal parity is $p$.
\paragraph{Recursion} When there are at least two priorities, let $p$ be the maximal parity. For concreteness, let us assume that $p$ is even.
Let us consider the sub-problem whether EVEN can force ODD to win a game with priority $p$ or whether ODD can force EVEN to cycle in states with priorities (strictly) smaller than $p$ (in the original game, ODD may win or lose, but if he loses, it will be with a parity smaller than $p$). This sub-problem can be cast as a mean payoff game with weight function:
\begin{align}
  \forall x,~ w(x) = \1_{p(x)=p}.
\end{align}
Indeed, writing $v$ the optimal value of this mean-payoff game,



Consider the finite $k$-horizon solution to this problem for $k=1,2,\dots$: starting with $v_0(x)=0$, we have
\begin{align}
  \forall x \in X_0, ~~v_{k+1}(x) & = w(x) + \max_{y;(x,y)\in E} v_k(y), \\
  \forall x \in X_1, ~~v_{k+1}(x) & = w(x) + \min_{y;(x,y)\in E} v_k(y).
\end{align}
It is well known that $\frac{v_k}{k}$ tends to $v$ when $k$ tends to $\infty$. As we are going to see, 
\begin{align}
 A &= \{~ x ~;~ v_N(x) \ge n ~\}\\
 B &= \{~ x ~;~ 0 < v_N(x) < n ~\}\\
 C & = \{~ x ~;~ v_N(x)=0 \}.
\end{align}
\begin{lemma}
  The infinite-horizon mean payoff game is won by EVEN on $A$ and by ODD on $B \cup C$. On an optimal play, none of the states $x \in B$ appears on a cycle. 
\end{lemma}

We recursively solve the parity game restricted to the set $C$, a game which only contains priorities (strictly) smaller than $p$, i.e. obtain for each node $x \in B$ its optimal parity $p_*(x)$. From this, we can propagate this optimal parity from $B$ to $A$ by iterating (at most $n$ times):
\begin{align}
  \forall x \in X_0 \cap A, ~~ p_*(x) &= \max_{y;(x,y)\in E} p_*(y), \\
  \forall x \in X_1 \cap A, ~~ p_*(x) &= \min_{y;(x,y)\in E} p_*(y),
\end{align}
where the max and min operators above use the order relation $\preceq$ on priorities:
\begin{align}
  p \prec p' \Leftrightarrow (-2)^{p}<(-2)^{p'}.  
\end{align}

As there is only one recursive call, and as the maximal priority necessarily decreases at each iteration, the above procedure takes at most $d$ iterations, and
\begin{theorem}
A parity game can be solved in polynomial time.
\end{theorem}


\bibliographystyle{plainnat}
\bibliography{biblio.bib} 

\end{document}
