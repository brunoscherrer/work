\documentclass{article}
\usepackage{fullpage}
\usepackage[utf8x]{inputenc}
\usepackage{amssymb}
\usepackage{amsthm}
\usepackage{amsmath}
\usepackage{hyperref}
\usepackage{cleveref}
\usepackage{autonum}
\usepackage{dsfont}

\newtheorem{theorem}{Theorem}
\newtheorem{lemma}{Lemma}
\newtheorem{corollary}{Corollary}
\newtheorem{remark}{Remark}


\title{Towards a strongly polynomial algorithm for deterministic payoff games ?}

\author{Bruno Scherrer\footnote{INRIA, Universit\'e de Lorraine, bruno.scherrer@inria.fr}}


\def\1{{\mathds 1}}
\def\N{\mathbb N}
\def\R{\mathbb R}
\def\={\stackrel{def}{=}}
\def\Xmax{X_{+}}
\def\Xmin{X_{-}}
\newcommand{\suc}[1]{E(#1)}


\begin{document}
\maketitle

\begin{abstract}
  Given a zero-sum two-player $\gamma$-discounted deterministic game with $n$ states, I try to build an algorithm that is polynomial on $n$ (and independent of $\gamma$). Here, I describe a contraction property that is independent of $\gamma$.
\end{abstract}

Consider a zero-sum two-player $\gamma$-discounted game with $n$ states and $m$ transitions, and its corresponding Bellman operators:
\begin{align}
  T_{\mu,\nu}v & = r + \gamma P_{\mu,\nu}v, \\
  T_{\mu}v & = \min_{\nu \in N} T_{\mu,\nu}v, \\
  \tilde T_{\nu} & = \max_{\mu \in M} T_{\mu,\nu}v, \\
  Tv & = \max_{\mu \in M} T_{\mu}v = \min_{\nu \in N} \tilde T_{\nu}v.
\end{align}
It is well-known that the optimal value $v_*$ is the only fixed point of $T$ and that any pair of stationary policies ($\mu_*,\nu_*)$ such that $T_{\mu_*,\nu_*}v_*=v_*$ form a pair of optimal policies.

We shall consider non-stationary policies $\vec\mu = (\mu_1,\dots,\mu_\ell) \in M^\ell$ and $\vec\nu = (\nu_1,\dots,\nu_\ell) \in N^\ell$. The operators above can be extended straightforwardly to this kind of policies:
\begin{align}
  P_{\vec\mu,\vec\nu} &= P_{\mu_1,\nu_1} \dots P_{\mu_\ell,\nu_\ell}, \\
  T_{\vec\mu,\vec\nu} v &= T_{\mu_1,\nu_1} \dots T_{\mu_\ell,\nu_\ell} v, \\
  T_{\vec\mu} v &= \min_{\vec\nu \in N^\ell} T_{\vec\mu,\vec\nu} v = T_{\mu_1} \dots T_{\mu_\ell} v, \\
  \tilde T_{\vec\nu} v &= \max_{\vec\mu \in M^\ell} T_{\vec\mu,\vec\nu} v = \tilde T_{\nu_1} \dots \tilde T_{\mu_\ell} v, \\
  T^\ell v & = \max_{\vec\mu \in M^\ell} T_{\vec\mu}v = \min_{\vec\nu \in N^\ell} \tilde T_{\vec\nu}v.
\end{align}
For any stationary policy $\mu$ or $\nu$, we shall write $\mu^\ell$ and $\nu^\ell$ for their non-stationary clones $(\mu,\mu,\dots,\mu)$ and $(\nu,\nu,\dots,\nu)$.

Let
\begin{align}
  I = \{~ (i,j) ~; 1 \le i \le j \le n ~\}.
\end{align}
For any non-stationary policies $\vec\mu = (\mu_1,\dots,\mu_n) \in M^n$ and $\vec\nu =(\nu_1,\dots,\nu_n) \in N^n$, for all $(i,j) \in I$, we shall write $\vec\mu_i^j$ and $\vec\nu_i^j$ for the sub-policies:
\begin{align}
  \vec\mu_i^j &= \mu_i \mu_{i+1} \dots \mu_{j-1} \mu_{j}, \\
  \vec\nu_i^j &= \nu_i \nu_{i+1} \dots \nu_{j-1} \nu_{j}.
\end{align}

Let
\begin{align}
  C = \{~ (x,c) ~;~ x \in X,~ 1 \le c \le n ~\}.
\end{align}
For all $(x,c) \in C$, consider the following sets of policies:
\begin{align}
  N_{x,c}(\vec\mu) & = \{~ \vec\nu \in N^c ~;~ \1_x P_{\vec\mu,\vec\nu} = \1_x \}, \\
  M_{x,c} &= \{~ \mu ~;~ \arg\min v_{\mu,\nu} \cap N_{x,c}(\mu^c) \neq \emptyset \}.
\end{align}
Observe that
\begin{align}
  \bigcup_{(x,c) \in C} M_{x,c} = M.
\end{align}
For all $(x,c) \in C$, consider the following thresholds:
\begin{align}
  v_{x,c} = \max_{\mu \in M_{x,c}} \min_\nu v_{\mu,\nu}(x).
\end{align}


\paragraph{An improvement step}

Assume we are given $\mu_k$. Let $v_k$ be the value of $\mu_k$ against its best adversary:
\begin{align}
v_k = T_{\mu_k} v_k = \min_{\vec\nu \in N^n} T_{\mu_k,\vec\nu} v_k.
\end{align}
We compute $\vec\mu$ and $\vec\nu \in N^n$ such that
\begin{align}
  T^n v = T_{\vec\mu,\vec\nu}v.
\end{align}

One can finally ``project'' the policy $\vec\mu \in M^n$ to a stationary policy $\mu_{k+1} \in M$ that is at least as good as $\vec\mu$ by choosing any policy $\mu_{k+1} \in M$ that satisfies
\begin{align}
T_{\mu_{k+1}} w_k = T w_k,
\end{align}
where
\begin{align}
  w_k &= \max_{1 \le \ell \le n} T_{\vec\mu_\ell^n} v_{\vec\mu}
\end{align}
and $v_{\vec\mu}$ is the value of $\vec\mu$ against its optimal opponent $\vec\nu'$:
\begin{align}
  v_{\vec\mu} &= T_{\vec\mu} v_{\vec\mu} = T_{\vec\mu,\vec\nu'} v_{\vec\mu} = v_{\vec\mu,\vec\nu'}.
\end{align}

\paragraph{Monotonicity}

One can see that $v_k \le T v_k$, by monotonicity of the operator $T$, we have
\begin{align}
  \label{croissance}
  T^n v_k \ge T^{n-1}v_k \ge \dots \ge T v_k  \ge v_k.
\end{align}
We know that $\vec\mu$ is better than $\mu_k$ since:
\begin{align}
  v_{\vec\mu} - v_k & =  v_{\vec\mu,\vec\nu'} - v_k \\
  & = (I-\gamma^n P_{\vec\mu,\vec\nu'})^{-1}(T_{\vec\mu,\vec\nu'} v_k - v_k) \\
  & = (I-\gamma^n P_{\vec\mu,\vec\nu'})^{-1}(\tilde T_{\vec\nu'} v_k - v_k) \\
  & \ge (I-\gamma^n P_{\vec\mu,\vec\nu'})^{-1}(\underbrace{T^n v_k - v_k}_{\ge 0}) \\
  & \ge 0.
\end{align}

\paragraph{``Strong'' contraction}

There necessarily exists a state $x$ and an integer $1 \le \ell \le n$, such that
\begin{align}
\1_x (P_{\vec\mu,\vec\nu'})^\ell = \1_x.
\end{align}
Now, there necessarily exist $(i,j) \in I$ and $y$ such that:
\begin{align}
  \1_x P_{\vec\mu_1^{i-1},\vec\nu_1^{i-1}} = \1_x P_{\vec\mu_1^j,\vec\nu_1^j} = \1_y = \1_y P_{\vec\mu_i^j,\vec\nu_i^j}.
\end{align}
Let $c=j-i+1$.

\begin{lemma}
  With the notations above,
\begin{align}
v_{\vec\mu}(x) - v_k(x) \ge \frac{1}{n^2} (v_{y,c} - v_k(y)).
\end{align}
\end{lemma}
\begin{corollary}
  We have the following contraction towards $v_{y,c}$:
  \begin{align}
    v_{y,c} - v_{k+1}(x) \le \left(1 - \frac{1}{n^2} \right)(v_{y,c}- v_k(x)).
  \end{align}
\end{corollary}

\begin{proof}
For the state $x$ mentionned above, we have:
\begin{align}
  v_{\vec\mu}(x) - v_k(x) & = \1_x (I-(\gamma^n P_{\vec\mu,\vec\nu'})^\ell)^{-1} (T^n v_k - v_k) \\
  & \ge \frac{1}{1-\gamma^{n\ell}} \1_x (T^n v_k - v_k)  = \frac{1}{1-\gamma^{n\ell}} \1_x (T_{\vec\mu,\vec\nu} v_k - v_k).
\end{align}


Write $w=T^{n-j}v_k$. By equation~\eqref{croissance}, we have
\begin{align}
  \1_x (T^n v_k - v_k) & \ge \1_y (T^{n-i-1} v_k  - T^{n-j} v_k) \\
  & = \1_y (T^c w  - w) \\
  & = \min_{\vec\nu'' \in N^c} \1_y (T_{\vec\mu_i^j,\vec\nu''}w-w) \\
  & = \min_{\vec\nu'' \in N_{y,c}(\vec\mu_i^j)} \1_y (T_{\vec\mu_i^j,\vec\nu''}w-w) \\
  & = \max_{\vec\mu' \in M^c} \min_{\vec\nu'' \in N_{y,c}(\vec\mu'} \1_y (\tilde T_{\vec\nu''}w-w)\label{eq1}
\end{align}
because of the choice of $y$.

Finally, observe that
\begin{align}
  v_{y,c} - v_k(y) & = \max_{\mu \in M_{x,c}} \min_{\nu'' \in N}  \1_y (v_{\mu,\nu''} - v_k) \\
  & = \max_{\mu \in M_{x,c}} \min_{\vec\nu'' \in N_{x,c}(\mu^c)}  \1_y (v_{\mu^c,\vec\nu''} - v_k) \\
  & = \max_{\mu \in M_{x,c}} \min_{\vec\nu'' \in N_{x,c}(\mu^c)}  \1_y (I-\gamma^c P_{\mu^c,\vec\nu''})^{-1} (T_{\mu^c,\nu''} v_k - v_k ) \\
  & = \max_{\mu \in M_{x,c}} \min_{\nu'' \in N_{x,c}(\mu^c)} \frac{1}{1-\gamma^c} \1_y(T_{\mu^c,\nu''} v_k - v_k ) \\
  & \le \max_{\vec\mu' \in M^c} \min_{\nu'' \in N_{x,c}(\vec\mu'} \frac{1}{1-\gamma^c} \1_y(T_{\mu^c,\nu''} v_k - v_k ).
  \label{eq2}
\end{align}
The result is obtained by combining Equations \eqref{eq1} and \eqref{eq2}.

\end{proof}


\bibliographystyle{plain}
\bibliography{biblio.bib} 

\end{document}
