\documentclass{article}
\usepackage{fullpage}
\usepackage[utf8x]{inputenc}
\usepackage{amssymb}
\usepackage{amsthm}
\usepackage{amsmath}
\usepackage{hyperref}
\usepackage{cleveref}
\usepackage{autonum}
\usepackage{dsfont}

\newtheorem{theorem}{Theorem}
\newtheorem{lemma}{Lemma}
\newtheorem{remark}{Remark}
\newtheorem{assumption}{Assumption}



\def\1{{\mathds 1}}
\def\N{\mathbb N}
\def\R{\mathbb R}
\def\={\stackrel{def}{=}}
\def\Xmax{X_{+}}
\def\Xmin{X_{-}}
\newcommand{\attr}[2]{\mbox{Attr}_{#1}(#2)}


\begin{document}

\section{Threshold procedure}

We consider here the following situation:
\begin{assumption}
  \label{sp}
    Both players can make sure that the parities by which games are won are smaller than $p$.
\end{assumption}
  
Let $i \equiv p \pmod{2}$ be the player that can win a game with parity $p$.
We want to compute the set $W_i$ for which Player $i$ can win with parity $p$ and its complementary $W_{1-i}$ for which he cannot (i.e. for which Player $1-i$ can force Player $i$ to get a parity stricly smaller than $p$).

For $j \in \{0,1\}$, consider the following set of vertices:
\begin{align}
  H_j &= \{~ x ~;~ p(x)>p\mbox{ and }p(x) \equiv j \pmod{2} ~\}, \\
  H & = H_i \sqcup H_j = \{~ x ~;~ p(x)> p ~\}.
\end{align}

We shall consider the game $G'$ on the set of vertices $V \backslash ( \attr{G,0}{H_0} \cup \attr{G,1}{H_1} )$. $G'$ only contains 2 differents parities, so we can solve it efficiently by Puri's reduction and determine optimal cycles that are won by both players, and deduce (through the attractor) the full winning regions.

Explain why this makes sense... if it does?

\end{document}
