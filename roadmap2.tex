\documentclass{article}
\usepackage{fullpage}
\usepackage[utf8x]{inputenc}
\usepackage{amssymb}
\usepackage{amsthm}
\usepackage{amsmath}
\usepackage{hyperref}
\usepackage{cleveref}
\usepackage{autonum}
\usepackage{dsfont}

\newtheorem{theorem}{Theorem}
\newtheorem{lemma}{Lemma}
\newtheorem{remark}{Remark}



\begin{document}

Bruno Scherrer, bruno.scherrer@inria.fr, June, 2022.
~\\

In this short note, I describe a roadmap in order to prove that a parity game can be solved in polynomial time along with an algorithm to do so.

Here is a simple procedure to find some winning region: let $p$ be the highest priority, $i$ the player.
First consider the question whether player $i$ can win with priority $p$ (this is a 2 priority problem). If yes. We have a winning.
If no, let $A$ be the set of states with priority $p$. Let $B$ be the $i$-attractor of $A$. If $B$ is the whole set of states, then we know that the game is won by $i$ with priority $p$. Otherwise, consider the game $G \backslash B$ (we know that it is won with priority stricly smaller than $p$). We repeat the procedure until one finds a $B$ that is the whole set of states: we thus have identified a set of states that are won with some priority $p'$.



\end{document}
