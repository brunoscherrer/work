\documentclass{article}
\usepackage{fullpage}
\usepackage[utf8x]{inputenc}
\usepackage{amssymb}
\usepackage{amsthm}
\usepackage{amsmath}
\usepackage{hyperref}
\usepackage{cleveref}
\usepackage{autonum}
\usepackage{dsfont}

\newtheorem{theorem}{Theorem}
\newtheorem{lemma}{Lemma}
\newtheorem{remark}{Remark}



\begin{document}

Bruno Scherrer, bruno.scherrer@inria.fr, April, 4th, 2022.
~\\

In this short note, I describe a roadmap in order to prove that a parity game can be solved in polynomial time along with an algorithm to do so.


\paragraph{1) Extension of parity game}

The goal is made more precise. Both players, if they lose, want to make the priority of the other player as low as possible. 
This precisely define an optimal priority. This problem can be reduced to MPG through a Puri like reduction.

\paragraph{2) Algorithm}

Let $p$ be the highest priority. Let $i$ be the corresponding player. Consider the parity game of which priorities smaller than $p$ are made neutral.
This can be translated to a mean payoff game with only two cost values: 0 and 1, which I believe can be partly ``solved'' efficiently (with at most $n^2$ steps of Value Iteration)

Consider the total value in this mean payoff game. Let
\begin{align}
  A &= \{ x ~;~ v(x)=\infty \} \\
  B &= \{ x ~;~ \Omega(x)=p \}
\end{align}
We then solve $G\(A \cup B)$ and obtain $p_*(x)$ for all $x$ not in $A \cup B$.
We know that $p_*(x)=p$ for all $x$ in A. It remains to propagate the information on the nodes for which we do not know the optimal priority ($B$, plus nodes that may 




\end{document}
