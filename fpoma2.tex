\documentclass[a4paper,11pt]{article}
\usepackage{amsmath,amssymb,amsthm}

\title{Algorithme polynomial pour les points fixes de fonctions max-affines}
\author{}
\date{}

\begin{document}

\maketitle

\section{Problème}

On considère un système de type point fixe défini par un opérateur max-affine sur $\mathbb{R}^n$ :
\begin{equation}\label{eq:pointfixe}
x_i = \max_{k \in K_i} [T^{(k)} x]_i, \quad i = 1,\dots,n,
\end{equation}
où pour tout $k$ et $y \in \mathbb{R}^n$, $T^{(k)}y$ est une fonction affine :
\begin{equation}
[T^{(k)} y]_i = \sum_{j=1}^n a_{ij}^{(k)} y_j + b_i^{(k)}.
\end{equation}

L'objectif est de déterminer l'ensemble $X^*$ des solutions de \eqref{eq:pointfixe}, éventuellement vide.

---

\section{Reformulation comme programme linéaire}

Considérons le problème d'optimisation suivant :
\begin{equation}\label{eq:lp}
\begin{aligned}
\min_{x \in \mathbb{R}^n} &\quad \max_{i,k \in K_i} \big(x_i - [T^{(k)} x]_i\big)\\
\text{sous les contraintes } &\quad x_i \ge [T^{(k)} x]_i, \quad \forall i,k.
\end{aligned}
\end{equation}

En introduisant une variable $z$, ce problème peut être écrit comme un programme linéaire classique :
\begin{equation}
\begin{aligned}
\min & \quad z \\
\text{sous les contraintes } & \quad x_i - [T^{(k)} x]_i \le z, \quad \forall i,k,\\
& \quad x_i - [T^{(k)} x]_i \ge 0, \quad \forall i,k.
\end{aligned}
\end{equation}

Ainsi, \eqref{eq:lp} peut être résolu efficacement avec des algorithmes de programmation linéaire.

---

\section{Algorithme pour déterminer $X^*$}

L'idée clé est d'utiliser l'optimisation pour identifier et supprimer les contraintes impossibles, ce qui conduit à un algorithme polynomial.

\begin{enumerate}
\item Initialiser les ensembles $K_i$.
\item Répéter :
\begin{enumerate}
    \item Résoudre le programme linéaire \eqref{eq:lp} et noter la solution optimale $(x^\dag, z)$ ainsi que l'indice $(i^\dag,k^\dag)$ correspondant à la contrainte la plus violée.
    \item Si $z = 0$, alors $x^\dag$ est une solution du système \eqref{eq:pointfixe} pour les ensembles $K_i$ restants. L'algorithme peut s'arrêter.
    \item Si $z > 0$, alors aucune solution $x^* \in X^*$ ne peut satisfaire la contrainte $(i^\dag,k^\dag)$ à l'égalité. On peut donc supprimer $k^\dag$ de $K_{i^\dag}$ sans perdre de solutions.
    \item Si pour un certain $i$, $K_i$ devient vide, alors $X^* = \emptyset$ (pas de solution).
\end{enumerate}
\item Répéter jusqu'à convergence.
\end{enumerate}

\paragraph{Remarques.} 
\begin{itemize}
\item L'algorithme s'arrête après au plus $\sum_i |K_i| - n$ étapes, car à chaque itération au moins un élément de $K_i$ est supprimé.
\item À la fin de l'élagage, l'ensemble $X^*$ est donné par toutes les solutions du système linéaire combinatoire :
\[
x_i = [T^{(k_i)} x]_i, \quad k_i \in K_i \text{ (ensembles réduits)}.
\]
\item Chaque étape consiste en la résolution d'un programme linéaire de taille $O(n \cdot \sum_i |K_i|)$, donc l'ensemble de l'algorithme est polynomial.
\end{itemize}

---

\section{Conclusion}

Cet algorithme fournit une procédure efficace pour déterminer l'ensemble de solutions de points fixes pour les fonctions max-affines. Il repose sur une combinaison de :

\begin{itemize}
\item reformulation comme programme linéaire,
\item identification des contraintes impossibles via l'optimisation,
\item élagage progressif des ensembles $K_i$.
\end{itemize}

La méthode garantit l'arrêt en nombre fini d'itérations et permet de conclure soit à l'existence, soit à l'absence de solutions.

\end{document}
