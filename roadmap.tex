\documentclass{article}
\usepackage{fullpage}
\usepackage[utf8x]{inputenc}
\usepackage{amssymb}
\usepackage{amsthm}
\usepackage{amsmath}
\usepackage{hyperref}
\usepackage{cleveref}
\usepackage{autonum}
\usepackage{dsfont}

\newtheorem{theorem}{Theorem}
\newtheorem{lemma}{Lemma}
\newtheorem{remark}{Remark}



\begin{document}

Bruno Scherrer, bruno.scherrer@inria.fr, April, 4th, 2022.
~\\

In this short note, I describe a roadmap in order to prove that a parity game can be solved in polynomial time along with an algorithm to do so.


\paragraph{1) Extension of parity game}

The goal is made more precise. Both players, if they lose, want to make the priority of the other player as low as possible. 
This precisely define an optimal parity. This problem can be reduced to MPG through a Puri like reduction.
Also we want to add a neutral priority (corresponding to 0 in the MPG reduction).

\paragraph{2) Algorithm}

Let $p$ be the highest priority. Let $i$ be the corresponding player. Consider the parity game of which priorities smaller than $p$ are made neutral.
This can be translated to a mean payoff game with only two cost values: 0 and 1, which I believe can be partly ``solved'' efficiently (with at most $n^2$ steps of Value Iteration)

Consider the total value in this mean payoff game. There are 3 kinds of states:
\begin{enumerate}
  \item states with infinite value (they are won by $i$)
\item states with finite values (they are won by $1-i$) among which:
  \begin{enumerate}
  \item states with value $0$
  \item states with value between $1$ and $n-1$.
  \end{enumerate}
\end{enumerate}
Starting from $v=0$, which is such that $v \ge Tv$, the value is non-decreasing.


Algorithm: Identify the three sets 1, 2-a and 2-b. Mark values of 1. Call recursively the algorithm on 2-a, propagate obtained values on 2-a. 

\end{document}
