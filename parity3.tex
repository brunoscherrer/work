\documentclass{article}
\usepackage{fullpage}
\usepackage[utf8x]{inputenc}
\usepackage{amssymb}
\usepackage{amsthm}
\usepackage{amsmath}
\usepackage{hyperref}
\usepackage{cleveref}
\usepackage{autonum}
\usepackage{dsfont}
\usepackage{natbib}


\newtheorem{theorem}{Theorem}
\newtheorem{lemma}{Lemma}
\newtheorem{remark}{Remark}
\newtheorem{corollary}{Corollary}

\title{A polynomial algorithm for the parity game}

\author{Bruno Scherrer\footnote{INRIA, Universit\'e de Lorraine, bruno.scherrer@inria.fr}}


\def\1{{\mathds 1}}
\def\N{\mathbb N}
\def\R{\mathbb R}
\def\Q{\mathbb Q}
\def\G{{\cal G}}
\def\v{\overline v}
\def\pa{Player~0}
\def\pb{Player~1}


\begin{document}
\maketitle

\begin{abstract}
  ...
\end{abstract}


For any integers $i \le j$, write $[i,j]$ for the set of integers $\{i,i+1,\dots,j\}$.

A parity game between two players, \pa{ }and \pb, can be described by a tuple $\G=(X=[1,n]=X_0 \sqcup X_1,~ E=[1,m],~ \Omega:X \to [1,d])$ with $(n,m,d) \in \N^3$.
$(X,E)$ is a directed graph. $X$ is a set of $n$ nodes and $E$ a set of $m$ directed edges such that each node has at least one successor node. 
The set of nodes  $X$ is partitioned into a set of nodes $X_0$ belonging to \pa{ }and a set of nodes $X_1$ belonging to \pb. The function $\Omega:X \to [1,d]$ assigns an integer label called priority to each node of the graph.
A play is an infinitely long trajectory $(x_0,x_1,\dots)$ generated from some starting node $x_0$: at any time step $t$, the player to which the node $x_t$ belongs chooses $x_{t+1}$ among any of the outgoing edges of $E$ starting from $x_t$. The winner of the game is decided from the infinte sequence of priorities $(\Omega(x_0),\Omega(x_1),\dots)$ occuring through the play: if the highest priority occurring infinitely often is even (resp. odd), then \pa{ }(resp. \pb) wins.

It is known (cf. \citet{zielonka98}) that there exist optimal strategies that are positional (i.e. that are mapping from nodes to outgoing edges). In particular, when both players follow these positional strategies from some node $x_0$, the play follows a (potentially empty) path followed by an infinitely-repeated cycle $(x_1,\dots,x_{c(x_0)})$ for some $c(x_0) \in [1,n]$.

The goal of this paper is to describe a polynomial algorithm for computing optimal strategies for both players.


\section{An incremental procedure}

Consider a parity game $\G$ that has at least two different priorities. Let $p$ be the maximal priority and let $i \equiv p \mod 2$ be the corresponding player. 
In this section, we shall consider the sub-problem whether Player $i$ can win the game with priority $p$ or whether Player $1-i$ can force Player $i$ to cycle in nodes with priorities (strictly) lower than $p$ (in $\G$, Player $1-i$ may win or lose, but if he loses, it will be with a priority lower than $p$).

Consider the total payoff game $\G'=(X,~ E,~ g)$ with cost function
\begin{align}
  \forall x,~ g(x) = (-1)^p \1_{\Omega(x)=p},
\end{align}
in which \pa{ }wants to maximize the total cost on induced infinitely-long trajectories while \pb{ }wants to minimize it.

Let $N$ and $M$ be the set of valid decision rules respectively for \pa{ }and \pb:
\begin{align}
  M = \{~ \mu:X_0 \to X ~;~ \forall x \in X_0,~ \mu(x) \in \{~ y;(x,y) \in E\} ~\},\\
  N = \{~ \mu:X_1 \to X ~;~ \forall x \in X_1,~ \mu(x) \in \{~ y;(x,y) \in E\} ~\}.
\end{align}
For any pair of decision rules $\mu \in M$ and $\nu \in N$, let $P_{\mu,\nu}$ be the corresponding transition matrix
\begin{align}
  \forall x \in X_0,~ \forall y \in X,~ P_{\mu,\nu}(x,y) &= \1_{y=\mu(x)}, \\
  \forall x \in X_1,~ \forall y \in X,~ P_{\mu,\nu}(x,y) &= \1_{y=\nu(x)}.
\end{align}
Consider the Bellman operators associated to this total payoff game:
\begin{align}
  T_{\mu,\nu}v &= g + P_{\mu,\nu}v,\\
%  T_{\mu}v & = \min_{\nu \in N} T_{\mu,\nu}v,\\
%  \tilde T_{\nu}v & = \max_{\mu \in M} T_{\mu,\nu}v,\\
  T v & = \max_{\mu \in M} \min_{\nu \in N}\tilde T_{\mu,\nu}v.
\end{align}
Consider the following functions
\begin{align}
  v_{\mu,\nu} &= \lim_{k \to \infty}(T_{\mu,\nu})^k0, \\
%  v_{\mu} &= \lim_{k \to \infty}(T_{\mu})^k0, \\
%  \tilde v_{\mu} &= \lim_{k \to \infty}(\tilde T_{\nu})^k0, \\
  v_* &= \lim_{k \to \infty}T^k0.
\end{align}
...

The motivation for introducing the above total payoff game is the following simple observation:
\begin{lemma}
  Player $i$ can win the parity game $\G$ with priority $p$ from some starting node $x$ if and only if $|v_*(x)|=\infty$.
\end{lemma}
\begin{corollary}
  Player $1-i$ can force Player $i$ to cycle on nodes with priorities lower than $p$ from some starting node $x$ if and only if $|v_*(x)|<\infty$.
\end{corollary}

To provide an efficient algorithm for parity games, we shall need to deepen the above observation.
For concreteness, assume from now on that the maximal priority $p$ is even (the odd case is similar). Therefore $i=0$ and for all $x$, $g(x) \in \{0,1\}$.

Let $Y$ be the set of nodes from which \pb{ }can prevent \pa{ }to cycle infinitely often in nodes with parity $p$:
\begin{align}
Y = \left\{~ x \in X~;~  v_*(x)<\infty  ~\right\} = \left\{~ x \in X~;~  \exists \nu \in N,~ \forall \mu \in M,~ v_{\mu,\nu}(x)<\infty  ~\right\}
\end{align}
Let $F$ be the set of positional strategies by which \pb{ }can prevent \pa{ }to cycle infinitely often in nodes with parity $p$ from some starting node:
\begin{align}
F = \left\{~ \nu \in N ~;~ \exists x \in X,~ \forall \mu \in M,~ v_{\mu,\nu}(x)<\infty ~\right\}.
\end{align}
Let $Z$ be the set of nodes from which \pb{ }only visits nodes with parity (strictly) smaller than $p$:
\begin{align}
Z = \left\{ x \in X~;~ v_*(x)=0  \right\}. \label{Z}
\end{align}

We shall provide an efficiently-computable characterization of the sets $Y$, $F$ and $Z$.
Indeed, the set of strategies $F$ and the set $Y$ can be characterized as follows:
\begin{lemma}
  We have
  \begin{align}
    F &= \{~ \nu \in N ~;~ \exists x \in X,~ \forall \mu \in M,~ \exists j \in [0,n],~ \exists y \in Z,~ \1_x (P_{\mu,\nu})^j = \1_y ~\},\\
    Y & = \{~ x \in X~;~ \exists \nu \in N ~;~ \forall \mu \in M,~ \exists j \in [0,n],~ \exists y \in Z,~ \1_x(P_{\mu,\nu})^j = \1_y ~\}.
  \end{align}
\end{lemma}
\begin{proof}
  The characterization of $F$ is a consequence of the fact that after $j$ steps for some $j \in [0,n]$, the trajectory induced by $\mu$ and $\nu$ has necessarily entered its limiting cycle, that necessarily has value 0 (otherwise it would contradict the definition of $F$). The characterization of $Y$ is a consequence of that of $F$.
\end{proof}
In the parity game literature, $Y$ is called the \emph{$1$-attractor set of $Z$}. Given $Z$, $Y$ and $F$ can be computed in $n$ steps as follows:

...

In particular, we can observe that $F$ is a rectangular set.

Eventually, it turns out that the set $Z$ can also be characterized in a simple way:
\begin{lemma}
  The set $Z$ is equal to
  \begin{align}
    Z' = \left\{~ x \in X ~;~ [T^n 0](x) = 0 ~\right\}.
  \end{align}
\end{lemma}
\begin{proof}
  First observe that the operator $T$ is monotone and $0 \le T 0$ (since costs are non-negative). Therefore, the sequence of functions $(T^k 0)_{k \ge 0}$ is non-decreasing. An immediate consequence is that $Z' \subset Z$.
  Let us prove now that $Z \subset Z'$. Assume that $v_*(x)>0$ for some $x$. Then this implies that Player 0 has a strategy that can obtain at least a cost of $1$ from $x$ (on the infinite play). Then, necessarily, \pa{ }can also obtain it in $j$ steps with $j \in [1,n]$, and thus $[T^n 0](x) \ge [T^j 0](x)>0$.
\end{proof}

\section{An algorithm for the parity game}

\paragraph{Terminal condition:} If the game $\G$ only contains only one priority $p$, then we know that for all nodes, the optimal parity is $q$.
\paragraph{Recursion} When the game $\G$ has at least two different priorities, let $p$ be the maximal priority and let $i \equiv p \mod 2$ be the corresponding player. 
Let us consider the sub-problem whether Player $i$ can win the game with priority $p$ or whether Player $1-i$ can force Player $i$ to cycle in nodes with priorities (strictly) lower than $p$ (Player $1-i$ may win or lose, but if he loses, it will be with a priority lower than $p$). This sub-problem can be cast as a mean payoff game $\G'=(X,~ E,~ w)$ with cost function:
\begin{align}
  \forall x,~ g(x) = (-1)^p \1_{p(x)=p}.
\end{align}


We recursively solve the parity game restricted to the set $A$, i.e. the game $\G \backslash (B \cup C)$, a game which only contains priorities (strictly) lower than $p$, i.e. obtain for each node $x \in A$ its optimal parity $p_*(x)$. From this, we can propagate ? this optimal priority from $A$ to $B$ by iterating (at most $n$ times):
\begin{align}
  \forall x \in X_0 \cap B, ~~ p_*(x) &= \max_{y;(x,y)\in E} p_*(y), \\
  \forall x \in X_1 \cap B, ~~ p_*(x) &= \min_{y;(x,y)\in E} p_*(y),
\end{align}
where the max and min operators above use the order relation $\preceq$ on priorities:
\begin{align}
  p \prec p' \Leftrightarrow (-2)^{p}<(-2)^{p'}.  
\end{align}
As there is only one recursive call, and as the maximal priority necessarily decreases at each iteration, the above procedure takes at most $d$ iterations, and
\begin{theorem}
A parity game can be solved in polynomial time.
\end{theorem}


\bibliographystyle{plainnat}
\bibliography{biblio.bib} 

\end{document}
